\documentclass[11pt]{article}

\usepackage{a4wide}
\usepackage{mathptm}
\usepackage{xspace}
\usepackage{amsmath}
\usepackage{graphicx}
% 代码样式
\usepackage{algorithm}


% 参考文献引用上标,但是这个是英文reference
% \usepackage[super,square]{natbib}
\usepackage{algpseudocode}
\usepackage{tikz}
\usepackage{tkz-graph}
\usetikzlibrary{shapes.misc, positioning}
\usepackage{listings}
\usepackage{color}
% 设定页边距
\usepackage[top=2.5cm,bottom=2.5cm,left=2.5cm,right=2.5cm]{ geometry}
\usepackage[colorlinks = true,
            linkcolor = blue,
            urlcolor  = blue,
            citecolor = blue,
            anchorcolor = blue]{hyperref}
\usepackage{svg}
%设置首行缩进
\usepackage{indentfirst}
\setlength{\parindent}{2em }
\setlength{\parskip}{0pt }
%段前段后距离设置
%页眉和页脚%页眉和页脚
\definecolor{mauve}{rgb}{0.88, 0.69, 1.0}
% 创建一个新的命令,控制
\makeatletter
\def\@cite#1#2{\textsuperscript{[{#1\if@tempswa , #2\fi}]}}
\makeatother


\lstset{frame=tb,
  language=Java,
  aboveskip=3mm,
  belowskip=3mm,
  showstringspaces=false,
  columns=flexible,
  basicstyle={\small\ttfamily},
  numbers=left,
  numberstyle=\tiny\color{gray},
  keywordstyle=\color{blue},
  commentstyle=\color{dkgreen},
  stringstyle=\color{mauve},
  breaklines=true,
  breakatwhitespace=true,
  tabsize=3
}

% Xelatex中文宏包
\usepackage{CTEX}
\begin{document}

\title{MarkLogic调研}

\author{@rh01 https://github.com/rh01}

\date{}

\maketitle

\begin{abstract}

  多模型数据库(Multi-Model Database)是新一代的数据库,与只支持单一数据模型的传统数据库有所不同,多模型数据库是一种在统一、
  综合的平台下同时支持多种不同的数据模型的数据库,这些数据模型可包括传统的关系模型和NoSQL数据模型,并且多模型数据库
  具有自己一种或多种的查询语言,并不仅仅依赖于传统的SQL查询语言,这使得数据组织、管理和操作变得十分的简单和便捷。
  在企业中,拥有结构良好的数据和基于NoSQL技术的综合数据平台对用户是有益的 \cite{lu2017multi} ,
  这种方法显著地降低了集成,迁移,开发,维护和运营等问题.
  因此,在本论文中,我们将介绍多数据模型的数据库代表-MarkLogic, 将会介绍它的底层架构,所支持的数据模型,以及在底层和顶层如何访问、管理和操作
  数据,最后设计Benchmark来对MarkLogic进行评测,并与文档数据库MongoDB进行比较.

  \noindent 关键字:Multi-Model Database; MarkLogic; MongoDB

\end{abstract}



%\input{commands}

\input{sections/introduction}

\input{sections/background}

\input{sections/api}

\input{sections/experiments}

\input{sections/conclusion}


\bibliographystyle{plain}
\bibliography{report}


\end{document}
